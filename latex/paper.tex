%%%%%%%%%%%%%%%%%%%%%%%%%%%%%%%%%%%%%%%%%%%%%%%%%%%%%%%%%%%%%%%%%%%%%%%%%%%%%%%%
% Template for USENIX papers.
%
% History:
%
% - TEMPLATE for Usenix papers, specifically to meet requirements of
%   USENIX '05. originally a template for producing IEEE-format
%   articles using LaTeX. written by Matthew Ward, CS Department,
%   Worcester Polytechnic Institute. adapted by David Beazley for his
%   excellent SWIG paper in Proceedings, Tcl 96. turned into a
%   smartass generic template by De Clarke, with thanks to both the
%   above pioneers. Use at your own risk. Complaints to /dev/null.
%   Make it two column with no page numbering, default is 10 point.
%
% - Munged by Fred Douglis <douglis@research.att.com> 10/97 to
%   separate the .sty file from the LaTeX source template, so that
%   people can more easily include the .sty file into an existing
%   document. Also changed to more closely follow the style guidelines
%   as represented by the Word sample file.
%
% - Note that since 2010, USENIX does not require endnotes. If you
%   want foot of page notes, don't include the endnotes package in the
%   usepackage command, below.
% - This version uses the latex2e styles, not the very ancient 2.09
%   stuff.
%
% - Updated July 2018: Text block size changed from 6.5" to 7"
%
% - Updated Dec 2018 for ATC'19:
%
%   * Revised text to pass HotCRP's auto-formatting check, with
%     hotcrp.settings.submission_form.body_font_size=10pt, and
%     hotcrp.settings.submission_form.line_height=12pt
%
%   * Switched from \endnote-s to \footnote-s to match Usenix's policy.
%
%   * \section* => \begin{abstract} ... \end{abstract}
%
%   * Make template self-contained in terms of bibtex entires, to allow
%     this file to be compiled. (And changing refs style to 'plain'.)
%
%   * Make template self-contained in terms of figures, to
%     allow this file to be compiled. 
%
%   * Added packages for hyperref, embedding fonts, and improving
%     appearance.
%   
%   * Removed outdated text.
%
%%%%%%%%%%%%%%%%%%%%%%%%%%%%%%%%%%%%%%%%%%%%%%%%%%%%%%%%%%%%%%%%%%%%%%%%%%%%%%%%

\documentclass[letterpaper,twocolumn,10pt]{article}
\usepackage{usenix2019_v3}

% to be able to draw some self-contained figs
\usepackage{tikz}
\usepackage{amsmath}

% inlined bib file
\usepackage{filecontents}

\usepackage{mdwlist}
\usepackage{times}
\usepackage{longtable}
\usepackage{multirow}
\usepackage{wrapfig}
\usepackage{amsmath, amsthm, amssymb}
\usepackage{amsmath}
\usepackage{wasysym} % moon symbols
\usepackage{amsfonts}
\usepackage{fancyhdr}
\usepackage{graphicx}
\iffalse
\usepackage[pdftex, colorlinks, 	
	    pdfstartview=FitH,
	    linkcolor=black,
	    citecolor=black,
	    urlcolor=black,
	    filecolor=black]{hyperref}
\fi
\usepackage{floatrow}
\usepackage{enumerate}
\usepackage{enumitem}
\usepackage{tabularx}
\definecolor{riceblue}{RGB}{0,32,91}

% Set heading fonts for Rice Blue
%\usepackage{sectsty}
%\chapterfont{\color{riceblue}}
%\sectionfont{\color{riceblue}}

\usepackage{xspace}
\usepackage{rotating}

\newcommand{\etal}{\emph{et al.}\xspace}
\newcommand{\eg}{\emph{e.g.,}\xspace}
\newcommand{\ie}{\emph{i.e.,}\xspace}
\newcommand{\etc}{\emph{etc.}\xspace}

% mark sections as potential cuttable, if set to true then
% don't display. Only use this if you cut the sections
% cleanly.
% never set when in the submittalbe draft
\newenvironment{possiblecut}{}{\ignorespacesafterend}
\newenvironment{sizeredux}{\expandafter\comment}{\expandafter\endcomment\ignorespacesafterend}
%\def\docuts
\ifdefined\docuts
  \renewenvironment{possiblecut}{\expandafter\comment}{\expandafter\endcomment\ignorespacesafterend}
\fi

% Displayed sections with \cancut distinctly in pdf.
% never set when in the submittalbe draft
%\def\showcuts
\ifdefined\showcuts
  \renewenvironment{possiblecut}{\color{red}<<<Potential Cut:}{\ignorespacesafterend}
\fi

\newcommand{\Code}[1]{\mbox{\texttt{#1}}}
      
% Describing tasks for specific grant activities
\newtheorem{task}{Task}
\newcommand{\problem}{\noindent\textit{[Problem] - }\xspace}
\newcommand{\sol}{\noindent\textit{[Proposed Solution] - }\xspace}
\newcommand{\prelim}{\noindent\textit{[Preliminary Work] - }\xspace}
\newcommand{\proposed}{\noindent\textit{[Proposed Work] - }\xspace}
\newcommand{\eval}{\noindent\textit{[Proposed Evaluation] - }\xspace}
\renewcommand{\problem}{{{\it \underline{Problem}}.}\xspace}
\renewcommand{\sol}{{{\it \underline{Proposed Solution}}.}\xspace}
\renewcommand{\prelim}{{{\it \underline{Preliminary Work}}.}\xspace}
\renewcommand{\proposed}{{{\it \underline{Proposed Work}}.}\xspace}
\renewcommand{\eval}{{{\it \underline{Proposed Evaluation}}.}\xspace}


% Comments for each person
\newif\ifcomments
\commentstrue
\ifcomments
  % Allow the comments
  %% review command -> for drawing attention to parts that are not ready yet
  \newcommand{\review}[1]{{\bf\textcolor{red}{[#1]}}}
  \newcommand{\notify}[1]{{\bf\textcolor{blue}{
    [COMMENT FOR EDITOR -- #1]
  }}}
  \newcommand{\notes}[1]{{\footnotesize\textcolor{green}{[NOTE: #1]}}}
  \newcommand{\writeme}[1]{{\textcolor{gray}{[#1]}}}
  %\newcommand{\todo}[1]{{{\it \color{gray}\{#1\}}}}

  %%% Comments, general-purpose macros, etc.
  \usepackage{xcolor}
  \definecolor{darkgreen}{rgb}{0,0.6,0}
  %\def\ndd#1{\textbf{\textsl{ $\langle\!\langle$\textcolor{blue}{NDD: #1}$\rangle\!\rangle$}} }
  \newcommand{\ndd}[1]{\fcolorbox{cyan}{white}{}\footnote{\textcolor{cyan}{[#1--ndd]}}}
  %\newcommandx{\ndd}[2][1=]{\todo[linecolor=blue,backgroundcolor=blue!25,bordercolor=blue,#1]{#2}}

  \newcommand{\topic}[1]{\textcolor{orange}{#1}}
  \newcommand{\structure}[1]{\noindent{\em\textcolor{darkgreen}{[#1]}}}
  \newcommand{\point}[1]{{\bf\textcolor{violet}{<#1>}}}
  \newcommand{\paroutline}[2]{\textbf{\structure{#1}\point{#2}}}
  \def\outline#1{}
  \newcommand{\secpurpose}[1]{{{\em\textcolor{gray}{Section Purpose: #1}}}}
\else
  % Do nothing
  \def\ndd[1]{}
  \def\review[1]{\errmessage{NO review macros when comments turned off!}}
    \def\notify[1]{}
    \def\notes[1]{}
    \def\writeme[1]{}
    \def\todo[1]{}
  \def\topic{}
  \def\structure{}
  \def\point{}
  \newcommand{\paroutline}[2]{}
  \def\secpurpose#1{}
\fi

\newcommand{\system}{MASS\xspace}
\newcommand{\systemfull}{My Awesome SyStem\xspace}


%don't want date printed
\date{}

% Title is at the end of header
% make title bold and 14 pt font (Latex default is non-bold, 16 pt)
\title{\Large \bf \system: My Cool Title}

\iffalse % get to add when accepted!!!
%for single author (just remove % characters)
\author{
{\rm Your N.\ Here}\\
Your Institution
\and
{\rm Second Name}\\
Second Institution
% copy the following lines to add more authors
% \and
% {\rm Name}\\
%Name Institution
} % end author
\else
\author{Anonymized \#31i73}
\fi

%%----------------------------------------------------------
\begin{document}
%%----------------------------------------------------------

\maketitle

%%----------------------------------------------------------
\begin{abstract}
%%----------------------------------------------------------

  % -------------------- Compelling context locating the work
  The context and compelling trend is\ldots
  % -------------------- But there's a problem, focusing the work
  Unfortunately, this leads to the problem of\ldots
  % -------------------- [OPTIONAL] Gap in lit
  Despite many attempts, existing approaches fail to\ldots 
  % -------------------- Response
  In this work we present, \system, that specifically does something other work
  doesn't do.
  % -------------------- Key insight/originality/innovation
  The core insight is\ldots
  % -------------------- Resuls/eval
  A prototype of \system has been implemented that demonstrates\ldots
  % -------------------- Argue its value and potential impact
  Overall, we belive this approach provides significant\ldots

%%----------------------------------------------------------
\end{abstract}
%%----------------------------------------------------------

%%----------------------------------------------------------
\section{Introduction}
%%----------------------------------------------------------

  % --------------------
  Here's a really great idea that cites some good
  research~\cite{Dautenhahn:Nested:2015}.

  % --------------------
  Problem + why it's important

  % --------------------
  Research context or gap in existing approaches

  % --------------------
  The aim/goal/hypothesis (typically a single sentence, but
     can be longer)

  % --------------------
  The proposed solution: key intuition on original contribution and solution

  % --------------------
  How you evaluated it and key results

  % --------------------
  Our core contributions include: 
  %
  \begin{itemize*}
      %
    \item Most important original contribution (described in
      some Section~\ref{sec:?}.
      %
    \item Second most important original contribution
      (described in some Section~\ref{sec:?}.
      %
    \item Third (and typically last) most important original
      contribution (described in some Section~\ref{sec:?}.
      %
  \end{itemize*}


%%----------------------------------------------------------
\section{Background/Problem/Motivation}
%%----------------------------------------------------------

  % --------------------
  A key problem has been exposed at a high level, in this
  section we describe relevant background, contextualize the
  problem further so you understand it and the solution
  space, and or further motivate our approach.


%%----------------------------------------------------------
\section{Design}
%%----------------------------------------------------------
  % --------------------
  \begin{figure}
    %
    \begin{center}
      % Replace with pdf fig
      \begin{tikzpicture}
        \draw[thin,gray!40] (-2,-2) grid (2,2);
        \draw[<->] (-2,0)--(2,0) node[right]{$x$};
        \draw[<->] (0,-2)--(0,2) node[above]{$y$};
        \draw[line width=2pt,blue,-stealth](0,0)--(1,1)
              node[anchor=south west]{$\boldsymbol{u}$};
        \draw[line width=2pt,red,-stealth](0,0)--(-1,-1)
              node[anchor=north east]{$\boldsymbol{-u}$};
      \end{tikzpicture}
    \end{center}
    %
    \caption{\label{fig:vectors} Each paper should have a killer
    figure 1 that captures the essence of the approach idea. It
    can happen that figure 1 is a great related work comparison
    and then it will require a second system architecture
    figure.}
  %
  \end{figure}

  % --------------------
  How do we acheive that lofty goal? 
  %
  Well of course by solving several unsolved challenges.
  %
  The section sub headings in this paragraph should roughly
  align with those core original contributions and highlight
  both the overall design and key technical challenges and
  solutions in doing so.


%%----------------------------------------------------------
\section{Evaluation}
%%----------------------------------------------------------

  % --------------------
  We started the work with some hypothesis/goal/thesis and
  we want to provide justification that is the truth.
  %
  In this section we evaluate the solution to prove its
  validity in addressing the core problem of this work.


%%----------------------------------------------------------
\section{Discussion and Future Work}
%%----------------------------------------------------------

  % --------------------
  Our results are very exciting, and in this section we
  discuss key aspects, highlighting potential for impact and
  value, and then conclude with future extensions.


%%----------------------------------------------------------
\section{Related Work}
%%----------------------------------------------------------
  
  % --------------------
  \begin{quote}
    %
    If I have seen further, it is by standing on the
    shoulders of giants---Sir Isaac Newton
    %
  \end{quote}

  % --------------------
  In this section we describe key related efforts and
  describe how our approach contributes to the discussion.


%%----------------------------------------------------------
\section{Conclusion}
%%----------------------------------------------------------
  
  % --------------------
  This paper introduced \system, a new technique for doing
  awesome.
  %
  It clearly is original because it showed how to do the
  awesome.
  % 
  Evidence/results indicate something critical about it.
  %
  The implications are something FIERCE!



%-------------------------------------------------------------------------------
\section{Footnotes, Verbatim, and Citations}
%-------------------------------------------------------------------------------

Footnotes should be places after punctuation characters, without any
spaces between said characters and footnotes, like so.%
\footnote{Remember that USENIX format stopped using endnotes and is
  now using regular footnotes.} And some embedded literal code may
look as follows.

\begin{verbatim}
int main(int argc, char *argv[]) 
{
    return 0;
}
\end{verbatim}

Now we're going to cite somebody. Watch for the cite tag. Here it
comes. Arpachi-Dusseau and Arpachi-Dusseau co-authored an excellent OS
book, which is also really funny~\cite{arpachiDusseau18:osbook}, and
Waldspurger got into the SIGOPS hall-of-fame due to his seminal paper
about resource management in the ESX hypervisor~\cite{waldspurger02}.

The tilde character (\~{}) in the tex source means a non-breaking
space. This way, your reference will always be attached to the word
that preceded it, instead of going to the next line.

And the 'cite' package sorts your citations by their numerical order
of the corresponding references at the end of the paper, ridding you
from the need to notice that, e.g, ``Waldspurger'' appears after
``Arpachi-Dusseau'' when sorting references
alphabetically~\cite{waldspurger02,arpachiDusseau18:osbook}. 

It'd be nice and thoughtful of you to include a suitable link in each
and every bibtex entry that you use in your submission, to allow
reviewers (and other readers) to easily get to the cited work, as is
done in all entries found in the References section of this document.

Now we're going take a look at Section~\ref{sec:figs}, but not before
observing that refs to sections and citations and such are colored and
clickable in the PDF because of the packages we've included.

%-------------------------------------------------------------------------------
\section{Floating Figures and Lists}
\label{sec:figs}
%-------------------------------------------------------------------------------


%---------------------------
\begin{figure}
\begin{center}
\begin{tikzpicture}
  \draw[thin,gray!40] (-2,-2) grid (2,2);
  \draw[<->] (-2,0)--(2,0) node[right]{$x$};
  \draw[<->] (0,-2)--(0,2) node[above]{$y$};
  \draw[line width=2pt,blue,-stealth](0,0)--(1,1)
        node[anchor=south west]{$\boldsymbol{u}$};
  \draw[line width=2pt,red,-stealth](0,0)--(-1,-1)
        node[anchor=north east]{$\boldsymbol{-u}$};
\end{tikzpicture}
\end{center}
\caption{\label{fig:vectors} Text size inside figure should be as big as
  caption's text. Text size inside figure should be as big as
  caption's text. Text size inside figure should be as big as
  caption's text. Text size inside figure should be as big as
  caption's text. Text size inside figure should be as big as
  caption's text. }
\end{figure}
%% %---------------------------


Here's a typical reference to a floating figure:
Figure~\ref{fig:vectors}. Floats should usually be placed where latex
wants then. Figure\ref{fig:vectors} is centered, and has a caption
that instructs you to make sure that the size of the text within the
figures that you use is as big as (or bigger than) the size of the
text in the caption of the figures. Please do. Really.

In our case, we've explicitly drawn the figure inlined in latex, to
allow this tex file to cleanly compile. But usually, your figures will
reside in some file.pdf, and you'd include them in your document
with, say, \textbackslash{}includegraphics.

Lists are sometimes quite handy. If you want to itemize things, feel
free:

\begin{description}
  
\item[fread] a function that reads from a \texttt{stream} into the
  array \texttt{ptr} at most \texttt{nobj} objects of size
  \texttt{size}, returning returns the number of objects read.

\item[Fred] a person's name, e.g., there once was a dude named Fred
  who separated usenix.sty from this file to allow for easy
  inclusion.
\end{description}

\noindent
The noindent at the start of this paragraph in its tex version makes
it clear that it's a continuation of the preceding paragraph, as
opposed to a new paragraph in its own right.


\subsection{LaTeX-ing Your TeX File}
%-----------------------------------

People often use \texttt{pdflatex} these days for creating pdf-s from
tex files via the shell. And \texttt{bibtex}, of course. Works for us.

%-------------------------------------------------------------------------------
\section*{Acknowledgments}
%-------------------------------------------------------------------------------

The USENIX latex style is old and very tired, which is why
there's no \textbackslash{}acks command for you to use when
acknowledging. Sorry.

%-------------------------------------------------------------------------------
\section*{Availability}
%-------------------------------------------------------------------------------

USENIX program committees give extra points to submissions that are
backed by artifacts that are publicly available. If you made your code
or data available, it's worth mentioning this fact in a dedicated
section.

%-------------------------------------------------------------------------------
\bibliographystyle{plain}
\bibliography{bib}

%%%%%%%%%%%%%%%%%%%%%%%%%%%%%%%%%%%%%%%%%%%%%%%%%%%%%%%%%%%%%%%%%%%%%%%%%%%%%%%%
\end{document}
%%%%%%%%%%%%%%%%%%%%%%%%%%%%%%%%%%%%%%%%%%%%%%%%%%%%%%%%%%%%%%%%%%%%%%%%%%%%%%%%

%%  LocalWords:  endnotes includegraphics fread ptr nobj noindent
%%  LocalWords:  pdflatex acks
